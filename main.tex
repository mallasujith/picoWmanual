\documentclass[journal]{IEEEtran}
%\documentclass[journal,a5paper]{IEEEtran}
\usepackage[a5paper, margin=10mm]{geometry}
\setlength{\headheight}{1cm} % Set the height of the header box
\setlength{\headsep}{0mm}     % Set the distance between the header box and the top of the text
\usepackage{gvv-book}
\usepackage{gvv}


\begin{document}
\bibliographystyle{IEEEtran}
\onecolumn

\newpage
%\twocolumn
\onecolumn


\numberwithin{figure}{section}
\numberwithin{table}{section} 
\numberwithin{equation}{section}
\iffalse
\renewcommand{\thefigure}{\theenumi}
\renewcommand{\thetable}{\theenumi}
\renewcommand{\theequation}{\theenumi}
\fi
\section{Pico W}
\begin{enumerate}[label=\arabic*.,ref=\theenumi]
%\begin{enumerate}[label=\thesubsection.\arabic*.,ref=\thesubsection.\theenumi]

\item Power on Pico W using USB cable. Connect RUN on pico W to GND. Keep pressing BOOTSEL while removing the RUN-GND wire from GND.Pico W is now ready to be flashed.
\item Login to termux-debian and execute the following commands.
\begin{lstlisting}
cd .platformio/packages
git clone https://github.com/earlephilhower/arduino-pico
cd arduino-pico
git submodule update --init
cd pico-sdk
git submodule update --init
cd ../tools
python3 ./get.py
git clone https://github.com/gadepall/afw/ide/piosetup/codes
cd codes
pio run
\end{lstlisting}
% Insert your image here
\begin{center}
    \includegraphics[width=0.6\textwidth, height=0.3\textheight]{installation/picowpinout.jpg}
    \vspace{0.5em} % Adjust the amount of space as needed
    \captionof{figure}{Pin Diagram} % Adds a centered caption under the image
\end{center}
\vspace{0.5em} % Adjust the amount of space as needed
\item Connect RUN on pico W to GND. Keep pressing BOOTSEL while removing the RUN-GND wire from GND.Pico W is now ready to be flashed.
\begin{lstlisting}
Download EtchDroid from playstore.Flash the uf2 file using EtchDroid.
\end{lstlisting}
\vspace{0.5em} % Adjust the amount of space as needed
\item The Onboard LED will start blinking.
\end{enumerate}

\subsection{Seven Segment}
\begin{enumerate}[label=\arabic*.,ref=\theenumi]
%\begin{enumerate}[label=\thesubsection.\arabic*.,ref=\thesubsection.\theenumi]
\item Execute the following code to drive the seven segment display.
\begin{lstlisting}
ide/sevenseg/codes/sevenseg/sevenseg.cpp
\end{lstlisting}
\item
Make connections according to Table \ref{table:ard_disp_num}
%\iffalse
\begin{table}[H]
\centering
\input{ide/sevenseg/figs/ard_disp_num}
\caption{}
\label{table:ard_disp_num}
\end{table}

%\fi
\item
Now generate the numbers 0-9 by modifying the above program.

\end{enumerate}

\subsection{7447}
\begin{enumerate}[label=\arabic*.,ref=\theenumi]
		\numberwithin{figure}{section}
\item
Now make the connections as per Table \ref{table:7447_ard}  and execute the following program to drive the seven segment display using 7447 IC. 
\begin{lstlisting}
ide/7447/codes/gvv_ard_7447/gvv_ard_7447.cpp
\end{lstlisting}

\begin{table}[H]
\centering
\input{ide/7447/figs/7447_ard.tex}
\caption{}
\label{table:7447_ard}
\end{table}

% 
\item
 $W,X,Y,Z$ are the inputs
and $A,B,C,D$ are the outputs. The code below realizes the Boolean logic for B, C and D.  Write the logic for A and verify.
\begin{lstlisting}
ide/7447/codes/inc_dec/inc_dec.ino
\end{lstlisting}

\item
Now make additional connections as shown in Table \ref{table:ip_7447_ard} and execute the following code.  Comment.
%			\lstinputlisting{ide/7447/codes/ip_inc_dec/ip_inc_dec.ino}
\begin{lstlisting}
ide/7447/codes/ip_inc_dec/ip_inc_dec.cpp
\end{lstlisting}

\solution
In this exercise, we are taking the number 5 as input to the pico W and displaying it on the seven segment display using the 7447 IC.
\begin{table}[H]
\centering
\input{ide/7447/figs/ip_7447_ard.tex}
\caption{}
\label{table:ip_7447_ard}
\end{table}
\item
Verify the above code for all inputs from 0-9.
\end{enumerate}
%\vspace{3em} % Adjust the amount of space as needed
\subsection{K-Map}
\begin{enumerate}[label=\arabic*.,ref=\theenumi]
%\begin{enumerate}[label=\thesubsection.\arabic*.,ref=\thesubsection.\theenumi]
\item Execute the code in
\begin{lstlisting}
ide/7447/codes/inc_dec/inc_dec.cpp
\end{lstlisting}
%
and modify it using the K-Map equations for $A,B,C$ and $D$. Execute and verify for each case.
\end{enumerate}

\subsection{7474}
\begin{enumerate}[label=\arabic*.,ref=\theenumi]
\item
Generate the CLOCK signal using the \textbf{blink} program in the pico W. 
\item
Connect the pico W, 7447 and the two 7474 ICs according to Table \ref{fig:ff_ard_pin} and Fig. \ref{fig:decade_counter}.

			\begin{table}[H]
%\begin{table}
\centering
\input{ide/7474/figs/ff_ard_pin.tex}
\caption{}
\label{fig:ff_ard_pin}
%\end{table}
\end{table}
%
\item
Intelligently use the codes in 
\begin{lstlisting}
ide/7447/codes/inc_dec/inc_dec.ino
\end{lstlisting}
and
\begin{lstlisting}
ide/7447/codes/inc_dec/ip_inc_dec.ino
\end{lstlisting}
to realize the decade counter in Fig. \ref{fig:decade_counter}.
% 
 \begin{figure}[H]
\begin{center}
\resizebox {0.5\columnwidth} {!} {
\input{ide/7474/figs/decade_counter.tex}
}
\end{center}
\caption{}
\label{fig:decade_counter}
\end{figure}
%
\end{enumerate}
\end{document}
